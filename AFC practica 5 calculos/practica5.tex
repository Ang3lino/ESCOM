\documentclass[12pt,letterpaper]{article}

\usepackage{graphicx}		%Paquete para anexar fotos
\usepackage[utf8]{inputenc}
\usepackage[english,spanish]{babel}
\usepackage{amsmath}		%Permite alinear ecuaciones
\usepackage{amsfonts}
\usepackage{amsthm}	%para poner el cuadrito qed
\usepackage{amssymb}
\usepackage[margin=2.5cm]{geometry}		%Establece el margen a 3cm
\usepackage[doublespacing]{setspace}		%	espaciado por linea de 1.5
%	\documentclass[twocolumn]{report}		%	pone dos columnas
	

\begin{document}

\section{problema}
	Por el divisor de voltaje, sabemos que:

	$V_{R_1} = \displaystyle \frac{E}{R_T}R_1 = 4.93$v \\
	$V_{R_2} = \displaystyle \frac{E}{R_T}R_2 = 2.31$v \\
	$V_{R_3} = \displaystyle \frac{E}{R_T}R_3 = 2.76$v\\

	así

	$V_1 = 10$v \\
	$V_2 = 5.07$v \\
	$V_3 = 2.76$v \\

\section{problema}

	Igualando lo propuesto y despejando para $R_3$
	\begin{align*}		
		V_3 &= (3k2) (I) \\
			&= \frac{(3k2)(10)}{3k2 + R_3} \\
			&= 5
	\end{align*}	
	Al despejar, $R_3 = 3k2$.	

\section{problema}
	
	$V_2 = 5v$, $V_3 = 3v$, $R_1 = R$

	Por ley de Ohm
	$I = \displaystyle \frac{E}{R_T}$ \\
	Aplicando divisor de voltaje

	$V_2 = \displaystyle \frac{E}{R_T}R = 5$ \\
	$V_3 = 10 - V_2 - \displaystyle \frac{E}{R_T}R_2 = 3$ \\

	Tenemos las ecuaciones

	$IR_2 = 2$ \\
	$IR = 5$ \\
	donde
	$R_2 = \displaystyle \frac{2}{5}R$ \\
	sustituyendo en
	$\displaystyle \frac{E}{R + \displaystyle \frac{2}{5}R + R_3} \displaystyle \frac{2}{5} R= 2$ \\ 
	vemos que $R_3 = \displaystyle \frac{3}{5} R$

\section {problema}
	Por ley de Ohm, sabemos
	$I_1 = \displaystyle \frac{V_s}{R_1} = 6m97A$ \\
	Sea $R_z = \left(\displaystyle\sum_{n=2}^{4} \frac{1}{R_n}\right)^{-1} = 434.21 \Omega $
	así, aplicando divisor de corriente, tenemos 

	$I_2 = \dfrac{I_1R_z}{R_2} = 3m03 A$ \\
	$I_3 = \dfrac{I_1R_z}{R_3} = 1m38 A$ \\
	$I_4 = \dfrac{I_1R_z}{R_4} = 918\mu18 A$


\end{document}

