\documentclass[12pt,letterpaper]{article}

\usepackage{graphicx}		%Paquete para anexar fotos
\usepackage[utf8]{inputenc}
\usepackage[english,spanish]{babel}
\usepackage{amsmath}		%Permite alinear ecuaciones
\usepackage{amsfonts}
\usepackage{amsthm}	%para poner el cuadrito qed
\usepackage{amssymb}
\usepackage[margin=3cm]{geometry}		%Establece el margen a 3cm

\providecommand{\abs}[1]{\lvert#1\rvert} %para usar el comando \abs{ } valor absoluto
% \providecommand{\norm}[1]{\lVert#1\rVert} %Para usar el comando \norm como la norma
\usepackage[doublespacing]{setspace}	%doble espaciado por linea

%\usepackage[onehalfspacing]{setspace}		%	espaciado por linea de 1.5
%	\documentclass[twocolumn]{report}		%	pone dos columnas


% \setlength{\parskip}{0.5cm plus 5mm minus 4mm}

\begin{document}

\begin{titlepage}
	\centering
	\includegraphics[width=0.15\textwidth]{ipn.PNG}\par\vspace{1cm}
	{\scshape\LARGE Escuela Superior de Cómputo \par}
	\vspace{1cm}
	{\scshape\Large Ensayo \par}
	\vspace{1.5cm}
	{\huge\bfseries ¿Soy feliz? \par}
	\vspace{2cm}
	{\Large\itshape por: \\ Ángel López Manríquez \\ \par}
	\vfill
	profesora\par
	Lic. Laura Lazcano Xoxotla 

	\vfill

% Bottom of the page
	{\large \today\par}
\end{titlepage}

\section{Introducción}

	¿Que es la felicidad? ¿acaso la felicidad radica en la ignorancia o es la ausencia de la tristeza? esto es un tema bastante filosofico, el cual  ha puesto a la humanidad desde que el hombre tiene conciencia de si mismo, pero, en si ¿que es la felicidad? la verdad es que no hay una definicion tautologica de el concepto de felicidad para todo ser humano. Para algunos la felicidad puede ser el ser millonario, el tener una estabilidad economica, tener salud, estar con las personas que han marcado un punto y aparte en nuestra vida... tal concepto no es compartido exactamente para dos personas distintas, y de esto es lo que se va a hablar en el ensayo.

\section{Desarrollo}
	
	El ser feliz es una emoción en donde la persona se siente bien consigo mismo durante algun periodo de tiempo y tal estado no tiene porque ser permanente, pero ¿que es lo que origina la felicidad al hombre? como ya se hizo mencion en la introducción no hay una forma excelente de saber que hizo feliz al hombre en algun instante de tiempo, sin embargo, es posible generalizar las causas de la felicidad para la mayoria de la gente, que son: tener una carrera finalizada, el tener una familia, una buena suma de dinero, el hecho de que el mundo sepa de que la persona es buena en algo, el estar conciente de uno mismo es util para algo, son algunos ejemplos de lo que hace feliz al ser humano, y es realmente impresionante lo que uno esta dispuesto a hacer para alcanzar tal estado, tan poderoso como una adicción, y es que es cierto, nadie quiere estar triste consigo mismo. Ya se han dado casos de lo que alguien esta dispuesto a hacer por este estado; casos tales como: el niño que hace la tarea para que sus padres le premien y este sea feliz por un periodo de tiempo, hasta casos extremos en donde personas roban para obtener el respeto de alguna organización delictiva, y es aqui donde se aprecia que los estados de felicidad pueden ser condicionados por la sociedad, tanto para bien como para mal, como personas que tuvieron una muy buena niñez y desean regresar a ese momento donde no tenian preocupaciones, o las personas que sufrieron a una temprana edad un trauma psicologico y les resulta dificil (si no que imposible) el ser feliz o al menos fingir una sonrisa. Como ya se ha probado en la psicologia, nadie es feliz o infeliz al nacer, todas las personas tienen su historia la cual es el nucleo de su estado de animo, en donde, uno como animal pensante y social deberia ser mas empatico cuando ve a alguien insatisfecho con lo que ha hecho con su vida y así, motivar a las personas y hacerles ver que valen por el hecho de ser humanos y hacerles ver que tienen la capacidad de hacer lo que quieran, aunque no culpo a la sociedad de no hacer esto siempre, pues, a veces las personas infelices se cierran ante el mundo, ante cualquier ayuda, argumentandose asi mismos que la vida es tal cual y no hay remedio.

	

	Personalmente, la felicidad esta conmigo por momentos, lo cual es normal, es imposible estar feliz todo el tiempo, aunque conozco personas que siempre andan con una sonrisa en el rostro y se rien por cualquier cosa, yo no... un chiste sin logica o coherencia para mi no tiene tiene gracia. No estoy diciendo que sea infeliz, al contrario, me siento satisfecho conmigo mismo de haber entrado a la ESCOM, del IPN, una de las carreras en donde cerca del \%12.5 es aceptado. Ademas de que es sabido que esta universidad te puede mandar a otras partes del mundo, a su vez de que me demostre a mi mismo (y a los demas) de que no soy inutil y que soy capaz de concluir una ingenieria en una de las mejores escuelas de México. Tambien, el hecho de gozar de salud fisica y mental hace que agradezca a la vida, pues, hay personas que no disponen de estas dos (cosa que no les impide ser felices). Como se aprecia, estos puntos personales son algunos de los cuales la felicidad esta compuesta.

\section {Conclusión}
	
	La felicidad es un estado que toda persona quiere conservar mediante acciones, ya sean buenas o malas, a su vez que es mas preferente enfocarse en las cosas positivas y hacer un esfuerzo por ser mas empaticos con las personas que nos rodean.

\end{document}	

