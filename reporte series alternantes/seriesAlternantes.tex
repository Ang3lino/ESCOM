\documentclass[12pt,letterpaper]{article}

\usepackage{graphicx}		%Paquete para anexar fotos
\usepackage[utf8]{inputenc}
\usepackage[english,spanish]{babel}
\usepackage{amsmath}		%Permite alinear ecuaciones
\usepackage{amsfonts}
\usepackage{amsthm}	%para poner el cuadrito qed
\usepackage{amssymb}
\usepackage[margin=2.5cm]{geometry}		%Establece el margen a 3cm
%\usepackage[]{geometry}		
\usepackage{fancyhdr} %encabezados y pie de pagina

\providecommand{\abs}[1]{\lvert#1\rvert} %para usar el comando \abs{ } valor absoluto
\providecommand{\norm}[1]{\lVert#1\rVert} %Para usar el comando \norm como la norma
%\usepackage[doublespacing]{setspace}	%doble espaciado por linea

\usepackage[onehalfspacing]{setspace}		%	espaciado por linea de 1.5
%	\documentclass[twocolumn]{report}		%	pone dos columnas

% \setlength{\parskip}{0.5cm plus 5mm minus 4mm}

\pagestyle{fancy}
\fancyhf{}
\rhead{Series alternantes}
\lhead{Ángel López Manríquez}
\rfoot{Pagina \thepage}

\begin{document}
\begin{titlepage}
	\centering
	\includegraphics[width=0.15\textwidth]{ipn.PNG}\par\vspace{1cm}
	{\scshape\LARGE Escuela Superior de Cómputo \par}
	\vspace{1cm}
	{\scshape\Large Reporte\par}
	\vspace{1.5cm}
	{\huge\bfseries Series alternantes \par}
	\vspace{2cm}
	{\Large\itshape por: \\ Ángel López Manríquez \\ \par}
	\vfill
	profesora\par
	Lic. Claudia Jisela Dorantes Villa

	\vfill

% Bottom of the page
	{\large \today\par}
\end{titlepage}

\section{Introducción}
	El concepto de serie se refiere a sumar sucesiones que siguen una determinada regla y se denota como $ \sum_{i=1}^{\infty} a_i $ o $ \sum a_i $ que es otra forma de decir $  a_1 + a_2 + \ldots $, en el presente reporte se abordaran a las series alternantes, que tienen la forma $ \sum_{k=1}^{\infty} b_k (-1)^k $ o , $ \sum_{k=1}^{\infty} b_k (-1)^{k + 1} $ con sus respectivos criterios de convergencia.
%	Salto de linea
%\cleardoublepage

\section{Desarrollo}
	Una serie alternante es una serie en la que los terminos consecutivos alternan de signo, por ejemplo:

	\[ 1-\frac{x^2}{2!}+\frac{x^4}{4!}+\cdots = \displaystyle \sum_{k=0}^{\infty} \frac{x^{2k}}		{(2k)!}(-1)^k = \cos x  \]

	\[ 1-1+1-1+\cdots = \displaystyle \sum_{k=0}^{\infty}(-1)^k \]
	
	Donde la primera es una serie de McLaurin para el coseno de $x$ y la segunda es la serie de Grandi.

	\subsection{Teorema de convergencia para series alternantes}
		Si la serie alternante 
			$$ \displaystyle \sum_{n=1}^{\infty} b_n (-1)^{n+1} $$
		cumple con 
			\begin{align*}
				\mbox{1) $b_{n+1} \leq b_n$} && \mbox{2) $\displaystyle \lim_{n \rightarrow \infty} b_n = 0$} 
			\end{align*}
		entonces la serie es convergente.	

	\subsection{Demostración}
		Por 1), tenemos
			$$ b_n - b_{n+1} \geq 0 $$
		Ahora, consideremos las sumas parciales como:
			\begin{align*}		
				S_2 &= b_1-b_2 \geq 0 \\
				S_4 &= b_1-b_2+b_3-b_4 = S_2+b_3-b_4 \geq S_2 &\mbox{Puesto que $b_3-b_4 \geq 0$}\\
				S_6 &= b_1-b_2+b_3-b_4+b_5-b_6 = S_4+b_5-b_6 \geq S_4 &\mbox{puesto que $b_5-b_6 \geq -0$}\\
				\vdots \\
				S_{2n} &= S_{2n-2}+b_{2n-1}-b_{2n} \geq S_{2n-2} &\mbox{puesto que $b_{2n-1}-b_{2n} \geq 0$}
			\end{align*}		
		Asi, vemos que $S_{2n}$ esta creciendo y esta siendo acotada por abajo por $S_{2n-2}$.
		
		Notemos que tambien podemos escribir la serie como:
			\begin{align*}
				S_{2n} &= b_1-b_2+b_3-b_4+b_5+\cdots-b_{2n-2}+b_{2n-1}-b_{2n}	\\
			 				 &= b_1-(b_2-b_3)-(b_4-b_5)+\cdots-(b_{2n-2}-b_{2n-1})-b_{2n}
			\end{align*}
		Donde cada diferencia indicada entre los parentesis es mayor o igual a cero y por la condicion 1) 			tambien sabemos que $a_{2n}$ es un termino positivo. Por lo que, podemos decir que $S_{2n} \leq b_1$ $\forall n$.		
		
		Asi, vemos que la serie esta acotada tanto por arriba como por abajo, por lo que la serie debe converger. Asumamos que su es $L$, es decir:
			$$ \displaystyle \lim_{n \rightarrow \infty} S_{2n} = L $$
		Tambien sabemos, por la condicion 2) que $\lim_{n \rightarrow \infty} b_n = 0$, por lo que, tenemos:
			\[ \displaystyle \lim_{n \rightarrow \infty} S_{2n+1} =  \displaystyle \lim_{n \rightarrow \infty} (S_{2n}+b_{2n+1}) = L + 0 = L \]
		Vemos que $S_{2n}$ y $S_{2n+1}$ convergen al mismo limite $L$ y sabemos que son secuencias de $S_n$, Por lo que la serie converge. 
		\qed	

		\subsection{series alternantes convergentes \\ \small{ejemplos}}
		
		\subsection{error en la aproximación}
			Algo que nos puede resultar util es aproximar una serie alternante (convergente, claro esta) si se cumplen 	los casos anteriores, sigamos asumiendo que $L = \displaystyle \lim_{n \rightarrow \infty} \sum_{k = 1}^{n} b_k(-1)^{k+1}$, podemos decir que $L \approx \displaystyle \sum_{k = 1}^{m} b_k (-1)^{k+1}$ o $L \approx S_m $ y 
			
				\begin{align*}		
					S_m \leq L \leq S_{m+1}	%&&\mbox{con m par}
				\end{align*}	
		
		Lamentablemente no se recurre mucho a esto, a menos que se estime la aproximacion de la suma total, El error involucrado al usar $L \approx S_m$ es el residuo $R_n = L - S_m$.
		
		\subsection{Teorema de estimación para series alternantes}
			Si la serie alternante 
				$$ \displaystyle \sum_{n=1}^{\infty} b_n (-1)^{n+1} $$
			cumple con 
			\begin{align*}
				\mbox{1) $b_{n+1} \leq b_n$} && \mbox{2) $\displaystyle \lim_{n \rightarrow \infty} b_n = 0$} \\			
				\mbox{entonces} && \abs{R_n} = \abs{L-S_n} \leq b_{n + 1}
			\end{align*}

		\subsection{demostración}
			Sabemos de la demostracion para la prueba de series alternantes que $L$ queda entre dos sumas parciales, por lo que se infiere que:
				$$ | L - S_n | \leq |S_{n+1} - S_n| = b_{n+1}	$$
				\qed


		\subsection{residuos en la aproximación de series alternantes \\ \small{ejemplos}}
		
	%\cleardoublepage

	\section{Conclusión}
		\subsection{Ángel}
			El estudio de las series resulta de gran importancia para las ciencias y la ingenieria, puesto 					que con las mismas es posible hacer integrales que con los metodos tradicionales es imposibles 
			resolverlas ($ \int e^{x^2} dx$, $ \int \sin (t^2)dt$, por mencionar algunos ejemplos). Asi como las 
			calculadoras, que mediante series de Taylor y McLaurin es posible la implementacion de las 
			funciones trigonometricas y logaritmos. 			
		
		%\subsection{Alan}
		
		
	\bibliography{referencias}


\end{document}